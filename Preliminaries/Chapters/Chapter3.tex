% Chapter Template

\chapter{ANALYSIS AND DESIGN} % Main chapter title

\label{Analysis And Design} % Change X to a consecutive number; for referencing this chapter elsewhere, use \ref{ChapterX}

\lhead{Chapter 3. \emph{Analysis And Design}} % Change X to a consecutive number; this is for the header on each page - perhaps a shortened title

%----------------------------------------------------------------------------------------
%	SECTION 1
%----------------------------------------------------------------------------------------

\hspace{30} In this chapter,   we   state   our   aim   of   contributing   to   the   BRL-­CAD   project  
and   explain   how   we   implemented   a   heart­shaped   primitive   in   the   project   design  
section.   Firstly,   we   introduce   the   concept   of   free   and   open   source   software. Secondly,   we   do   an   overview   of   the   BRL-­CAD   software   package.   After,   we  state   our   aim   of   contributing   to   the   BRL-­CAD   project.   Finally,   we   give   a   detailed  explanation   of   the   design   which   we   employed   to   implement   the   heart-­shaped primitive for BRL-­CAD.

%------------------------------------------------------------------------------------------------------------------------------
%		OPEN SOURCE COMMUNITY
%------------------------------------------------------------------------------------------------------------------------------
\section{The Open Source Community}

\hspace{30} Depending   on   how   we   choose   to   call   it,   \textit{Free/Libre/Open   Source  
Software   (FLOSS)},   \textit{Free   and   Open   Source   Software   (FOSS)}   or   simply   \textit{Open  
Source   Software   OSS)}   is   software   for   which   users   have   access   to   both   the  
source   code   and   binary   executables   and   is   licensed   under   a   license   which  
permits   its   users   to   read,   edit   and   distribute   the   software   to   anyone   and   for   any  
reason.   This   distinguishes   open   source   software   from   commercial   software  
which   is   distributed   by   giving   away   its   binary   executable   version   only.   Usually,  
OSS   is   distributed   at   no   cost   with   limited   restrictions   on   how   it   can   be   used.  
According   to   Eric   S.   Raymond[34],   one   of   the   most   prominent   evangelists   of  
the   open   source   movement,   hackerdom   can   be   likened   to   what   anthropologists  
call   a   gift   culture–   a   culture   wherein   members   gain   status   and   reputation   by  
giving   away   their   time,   creativity   and   skills   to   reading,   writing   and   debugging  
software,   publishing   useful   information   in   blogs   or   documents   like   Frequently  
Asked   Questions   (FAQs)   lists   as   well   as   handling   unglamorous   tasks   like  
maintaining   mailing   lists,   moderating   news   groups,   etc   without   any   monetary  
compensation.   The   word   \textit{hacker}   was   coined   by   a   shared   community   of   expert  
programmers   and   networking   masters   which   traces   its   history   back   to   the   days  
of   the   earliest   ARPAnet   experiments   and   time­sharing   minicomputers   who  
made   the   unix   operating   systems   and   the   world­wide   web   work.   As   opposed   to  
hackers,   \textit{crackers}   who   are   more   interested   in   breaking   software   and   perturbing  
phone systems.   

\hspace{30} Today,   the   open   source   community   has   become   a   self­organizing  
collaborative   social   network   of   hackers   driven   by   a   passion   to   solve   problems  
using   free   software   with   thousands   of   projects   hosted   on   Sourceforge[35]   and  
Github[36].   It   has   singularly   developed   some   software   packages   and   tools  
which   are   the   best   in   the   world   such   as   the   firefox   web   browser,   Apache  
web­server,   Linux   operating   systems   like   BSD,   Ubuntu,   Debian,etc,   the   MySQL  
database   management   system,   the   VLC   media   player,   programming  languages   and   tools   like   gcc,   C,   Perl,   Python,   Java,   etc   and   much   more.   Some  Examples   of   CAD   packages   within   the   open   source  community   include   BRL­CAD,   Blender,   FreeCAD,   openSCAD   and   LibreCAD,  
etc.

%--------------------------------------------------------------------------------------------------------------------------------

%--------------------------------------------------------------------------------------------------------------------------------
%	ANALYSIS OF THE WORK
%--------------------------------------------------------------------------------------------------------------------------------
\section{Analysis Of The Work}

\hspace{30} BRL­-CAD   (   pronounced   Be­-Are-­El-­CAD)   was   originally   conceived   and  
written   by   the   late   Mike   Muss,   a   programmer   and   networking   expert   who   also  
wrote   the   popular   PING   network   program.   In   1979,   the   United   States   Army's  
Ballistic   Research   Laboratory   (BRL)   (the   agency   responsible   for   creating  
ENIAC,   the   world's   first   general­purpose   electronic   computer   in   the   1940s)  
identified   a   need   for   tools   that   could   assist   with   the   computer   simulations   and  
analysis   of   combat   vehicle   systems   and   environments.   When   no   existing   CAD  
package   was   found   to   be   adequate   for   this   specialized   purpose,   Mike   and  
fellow   software   developers   began   developing   and   assembling   a   unique   suite   of  
utilities   capable   of   interactively   displaying,   editing,   and   interrogating   geometric  
models.   Those   early   efforts   subsequently   became   the   foundation   on   which  
BRL­CAD was built.  

\hspace{30} The   initial   architecture   and   design   of   BRL-­CAD   began   in   1979   and   its  
development   as   a   unified   software   package   kicked   off   in   1983   with   its   first  
public   release   the   following   year.   As   a   software   package   with   a   mature   code  
base   which   has   been   actively   developed   for   decades,   BRL-­CAD   pays   close  
attention   to   design   and   maintainability.   Like   other   FLOSS   packages,  
BRL­-CAD's   source   code   and   most   of   its   project   data   are   stored   in   a  
subversion   version   control   system   for   change   tracking,   collaborative  
development   and   is   redistributed   as   free   and   open   source   software   under   the  
Open   Source   Initiative   license   terms.   The   design   of   its   system   architecture   is  
based   on   a   unix­methodology   of   command   of   the   command­line   services,  
providing   many   tools   that   work   in   harmony   to   complete   a   specific   task.   These  
tools   include   geometry   and   image   converters,signal   and   image   processing  
tools,   various   ray   tracing   applications,geometry   manipulators,   and   much   more.  
They   will   also   be   used   to   test   that   the   geometric   properties   of   the   heart­shaped  
primitive works as we will see in Chapter Four.

\hspace{30} The   basic   layout   of   its   code   places   public   API   headers   in   the   top­level  
\textit{\textbf{include/}}   directory   and   source   code   for   both   applications   and   libraries   in   the  
\textit{\textbf{src/}}   directory.   The   following   is   a   partial   listing   of   how   BRL-­CAD's   source   code  
is organised in a typical checkout or source distribution. 

\textbf{Applications and Resources}  

\begin{itemize} 
\item✦ \textit{\textbf{db/}} for Example Geometry.  
\item✦ \textit{\textbf{doc/}} for project Documentation.  
\item✦ \textit{\textbf{include/}} for Public API headers.  
\item✦ \textit{\textbf{regress/}} for Regression test scripts  
\item✦ \textit{\textbf{src/}} for Application and library source.  
\item✦ \textit{\textbf{src/conv}} for Geometry converters.  
\item✦ \textit{\textbf{src/fb}} for Displaying data in windows.  
\item✦ \textit{\textbf{src/mged}} for the Multi­device geometry editor, the main GUI application.   
\item✦ \textit{\textbf{src/rt}} for Ray tracing applications.  
\item✦ \textit{\textbf{src/util}} for Image processing utilities.
\end{itemize}  

\textbf{Libraries}
  
\begin{itemize} 
\item✦ \textit{\textbf{src/libbn}} for Numerics library.  
\item✦ \textit{\textbf{src/libbu}} for Utility library.  
\item✦ \textit{\textbf{src/libgcv}} for Geometry conversion library.  
\item✦ \textit{\textbf{src/libged}} for Geometry Editing library.  
\item✦ \textit{\textbf{src/icv}} for Image conversion library.  
\item✦ \textit{\textbf{src/libpkg}} for Network Package library.  
\item✦ \textit{\textbf{src/librt}} for Ray­tracing library.  
\item✦ \textit{\textbf{src/libwbd}} for Geometry creation library.
\end{itemize}

\hspace{30} The   majority   of   BRL-­CAD's   source   code   is   written   in   ANSI/POSIX   C   with   the  
intent   of   strictly   conforming   with   the   C   standard.   The   core   libraries   are   all   C   API  
though   several   such   as   the   Utility   and   Ray­tracing   libraries   use   C++   for  
implementation   details.   Major   components   of   the   system   are   written   in   C,   C++,  
Tcl/Tk,   Bash   and   Php   with   source   code   files   using   extensions   such   as   *.c,   *.h,  
*.cpp,   *.tcl,   *.tk,   *.sh   and   *.php.   BRL­CAD   uses   the   CMake   build   system   for  
compilation and an inbuilt testing infrastructure in regress/ for unit testing.  

\hspace{30} BRL-­CAD   has   a   long­lasting   heritage   of   maintaining   verifiable,   validated  
and   repeatable   results   in   critical   portions   of   the   software   package,   particularly  
within   the   ray   tracing   library.   It   has   an   inbuilt   testing   infrastructure   which  
compares   all   program   output   against   benchmark   results   during   each   build.   The  
ray   tracing   library   is   a   multi­representational   library   which   lies   at   the   heart   of  
BRL­-CAD   and   uses   a   suite   of   other   libraries   for   other   basic   application  
functionality.   Considerable   attention   is   put   into   verification   and   validation  
throughout   the   package   which   includes   regression   tests   that   compare   runtime  
behaviour   against   known   results   and   reports   any   adverse   variances   from  
standard results as failures.  

%-----------------------------------
%	SUBSECTION 2
%-----------------------------------

\subsection{Subsection 2}
Morbi rutrum odio eget arcu adipiscing sodales. Aenean et purus a est pulvinar pellentesque. Cras in elit neque, quis varius elit. Phasellus fringilla, nibh eu tempus venenatis, dolor elit posuere quam, quis adipiscing urna leo nec orci. Sed nec nulla auctor odio aliquet consequat. Ut nec nulla in ante ullamcorper aliquam at sed dolor. Phasellus fermentum magna in augue gravida cursus. Cras sed pretium lorem. Pellentesque eget ornare odio. Proin accumsan, massa viverra cursus pharetra, ipsum nisi lobortis velit, a malesuada dolor lorem eu neque.

%----------------------------------------------------------------------------------------
%	SECTION 2
%----------------------------------------------------------------------------------------

\section{Main Section 2}

Sed ullamcorper quam eu nisl interdum at interdum enim egestas. Aliquam placerat justo sed lectus lobortis ut porta nisl porttitor. Vestibulum mi dolor, lacinia molestie gravida at, tempus vitae ligula. Donec eget quam sapien, in viverra eros. Donec pellentesque justo a massa fringilla non vestibulum metus vestibulum. Vestibulum in orci quis felis tempor lacinia. Vivamus ornare ultrices facilisis. Ut hendrerit volutpat vulputate. Morbi condimentum venenatis augue, id porta ipsum vulputate in. Curabitur luctus tempus justo. Vestibulum risus lectus, adipiscing nec condimentum quis, condimentum nec nisl. Aliquam dictum sagittis velit sed iaculis. Morbi tristique augue sit amet nulla pulvinar id facilisis ligula mollis. Nam elit libero, tincidunt ut aliquam at, molestie in quam. Aenean rhoncus vehicula hendrerit.
