% Chapter 2

\chapter{LITERATURE REVIEW OF GEOMETRIC MODELING} % Main chapter title

\label{Literature Review} % Change X to a consecutive number; for referencing this chapter elsewhere, use \ref{ChapterX}

\lhead{Chapter 2. \emph{Literature Review}} % Change X to a consecutive number; this is for the header on each page - perhaps a shortened title

%----------------------------------------------------------------------------------------
%      INTRODUCTORY PARAGRAPHS
%----------------------------------------------------------------------------------------
\hspace{30} With the advent of computers which could perform millions of floating
point operations in unit time and which are still growing faster, researchers who
believed computers could aid the processes of mechanical design and
manufacturing were faced with a critical issue – how to represent physical
reality using computer software. They sought the best data structures to
represent this reality and the most appropriate algorithms to manipulate these representations.

\hspace{30} BRL-­CAD supports a wide variety of geometric representations including an 
extensive set of traditional implicit primitive shapes as well as explicit primitives
made from collections of uniform B­spline surfaces, non­uniform rational B­spline (NURBS)
surfaces, n­on-manifold   geometry   (NMG)   and purely faceted polygonal mesh geometry.
Consequently, in this chapter, we review the existing work done by scholars in the 
field of geometric modeling which have been applied to the development of BRL-­CAD. 
First of all, it introduces the issue of representation and the notion of representation
 schemes.Then, it summarizes developments in wireframe modeling, surface modeling, solid
modeling and non­-manifold modeling (aka non­manifold geometry or nmg for short) with a keen
 eye on the algorithms underlying them.

\hspace{30} As we progress in our literature review from older forms of geometric
modeling to newer ones, we will discover that representation schemes were
closely linked to algorithmic efficiency and that it has always been normal to
expect designers to switch to newer ones in response to the improvements in
algorithmic performance.Despite these enhancements in algorithmic efficiency
within the designer community, we cannot say with complete certainty whether
traditional representation schemes can be relegated to the background. We
can only conclude that old and new representation paradigms co­exist and that
research led to representation schemes which supplemented the repertoire of
geometric modeling.  

%-----------------------------------------------------------------------------------------

%----------------------------------------------------------------------------------------
%	SECTION 1
%----------------------------------------------------------------------------------------

\section{Representation Schemes}

A representation $\textbf{\mathfrak{R}}$ of a solid or representation for short is a subset of
three­-dimensional Euclidean space denoted $\mathbb{E}^3$ which models a physical solid.  
According to [5], Requicha and Tilove stated that point set topology provided a
formal language for describing the geometric properties of solids and they also  
threw more light on the mathematical characteristics of solids such as a solid's
interior, boundary, complement, closure, boundedness and regularity.
Requicha [4] insisted that to be computationally useful, a representation should  
formally capture the following properties ;

\begin{itemize}
\item \textit{\textbf{Rigidity:}} Representations should have an invariant configuration
irrespective of their location and orientation.
\item \textit{\textbf{Homogeneity:}} A representation should have an interior.
\item \textit{\textbf{Finiteness:}} A representation must occupy a finite amount of space.
\item \textit{\textbf{Boundary   determinism:}} A representation must unambiguously determine
the interior of that solid.
\item \textit{\textbf{Closure}}: Representations of solids which are manipulated by rigid
motions and regularized boolean operations should produce representations of solids too.
\end{itemize}
These formal characteristics leave representations no choice than to be
bounded, closed, regularized and semi­analytic, hence their coinage r­sets
according to [5]. An \textit{\textbf{r-­set}} is simply a regular and bounded set in $\mathbb{E}^3$.

\hspace{30} A representation scheme is simply a relation between physical solids and their representations which can be characterized by the following properties;
\begin{itemize}
\item \textit{\textbf{Domain}}: A representation scheme must represent quite a number of useful geometric solids. 
\item \textit{\textbf{Unambiguity}}: A representation scheme should produce representations which intuitively capture the properties of the physical solid so that it can be easily distinguished from other representations.
\item \textit{\textbf{Uniqueness}}: A representation scheme should uniquely represent a solid object within a software's database.  
\item \textit{\textbf{Validity}}: Representation schemes should yield representations of solids which do not exist or are valid.  
\item \textit{\textbf{Closure}}: A Representation scheme which transforms (reflects, scales, rotates) a representation should yield other representations too.
\item \textit{\textbf{Compactness}}: Representation schemes should yield representations which save space and allow efficient algorithms to   determine desirable physical characteristics.
\end{itemize}
%--------------------------------------------------------------------------------------------------
%	WIREFRAME MODELING
%-----------------------------------------------------------------------------------------------------
\section{Wireframe Modeling}

For rectilinear objects whose edges are straight lines and whose faces are planar, the ordered pair of vertices
 $\textbf{\mathfrak{V}} \in \mathbb{E}^3$ and edges $\textbf{\mathfrak{E}} \in \mathbb{E}^3$ denoted by $(\textbf{\mathfrak{V}} , \textbf{\mathfrak{E}})$ is the object's wireframe.  
In a practical sense, it is the skeleton of an object wherein joints are vertices
and bones are edges. In [6], a six ­step algorithm to generate an object's
wireframe was developed wherein an object's wireframe was represented by a vertex table and an edge table. 
Although the work in [6] had drawbacks such as not checking the validity of input data, wireframe modeling has always provided designers with a chance to experiment with the final result of their models through sketching and it is frequently used to preview complex models. However, the use of only edge information left wireframe models ambiguous
on rectilinear polyhedra talk less of topological ones. Figure 2 below shows the wireframe of a sphere in greyscale.

%-----------------------------------
%	SUBSECTION 2
%-----------------------------------

\subsection{Subsection 2}
Morbi rutrum odio eget arcu adipiscing sodales. Aenean et purus a est pulvinar pellentesque. Cras in elit neque, quis varius elit. Phasellus fringilla, nibh eu tempus venenatis, dolor elit posuere quam, quis adipiscing urna leo nec orci. Sed nec nulla auctor odio aliquet consequat. Ut nec nulla in ante ullamcorper aliquam at sed dolor. Phasellus fermentum magna in augue gravida cursus. Cras sed pretium lorem. Pellentesque eget ornare odio. Proin accumsan, massa viverra cursus pharetra, ipsum nisi lobortis velit, a malesuada dolor lorem eu neque.

%----------------------------------------------------------------------------------------
%	SECTION 2
%----------------------------------------------------------------------------------------

\section{Main Section 2}

Sed ullamcorper quam eu nisl interdum at interdum enim egestas. Aliquam placerat justo sed lectus lobortis ut porta nisl porttitor. Vestibulum mi dolor, lacinia molestie gravida at, tempus vitae ligula. Donec eget quam sapien, in viverra eros. Donec pellentesque justo a massa fringilla non vestibulum metus vestibulum. Vestibulum in orci quis felis tempor lacinia. Vivamus ornare ultrices facilisis. Ut hendrerit volutpat vulputate. Morbi condimentum venenatis augue, id porta ipsum vulputate in. Curabitur luctus tempus justo. Vestibulum risus lectus, adipiscing nec condimentum quis, condimentum nec nisl. Aliquam dictum sagittis velit sed iaculis. Morbi tristique augue sit amet nulla pulvinar id facilisis ligula mollis. Nam elit libero, tincidunt ut aliquam at, molestie in quam. Aenean rhoncus vehicula hendrerit.
